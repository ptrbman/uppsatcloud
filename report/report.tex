\documentclass{article}

\usepackage[papersize={8.27in,11.69in},hmargin=0.5in,tmargin=1.055in,bmargin=1.055in,includehead]{geometry}
% paper size if 6in x 9in which is standard international size
% margin from top is 0.7in including header
% margin from bottom is 0.6in including footer
% book of two sided pages is specified, with horizontal margin of 0.6in
% binding offset is an additional 0.1in from centre fold.
% REAL textwidth is therefore 6 - 2*0.6 -0.1 = 4.7in = 11.938cm, 0.45\textwidth = 5.3721 cm with 3 margins of 0.3979cm
\usepackage{multicol}
\setlength{\columnsep}{0.27in}
%
\usepackage[
    %backend=bibtex,
    backend=biber,
    natbib=true,
    style=numeric,
    sorting=none,
    backref=true
]{biblatex} %,style=verbose-trad2
%\addbibresource{bib.bib}
\bibliography{bibliography}


\usepackage[explicit]{titlesec}
\usepackage{titletoc}
\usepackage{lmodern}
\usepackage{epigraph}
\usepackage{xpatch}  % All above used in titling
%
\usepackage{tikz}
%
\usepackage{pgfplots} % for better precision plots
\pgfplotsset{compat=1.10} % for newer version?
\usepgfplotslibrary{fillbetween} % for shading pgfplots

\usepackage{graphicx}
\usepackage{wrapfig} % for odd wrapping of text around figure
%
\usepackage{textcomp} % for currency %\usepackage[gen]{eurosym}
%
\usepackage{lipsum} %for filler text only
%
%\usepackage{bold-extra} % Used in Game Theory Section for bold sc's
\usepackage[T1]{fontenc}
%
\usepackage{mathtools}
\usepackage{amsthm}
\usepackage{bm} %for bold greeks
\usepackage{booktabs} %for better standard tables
\usepackage{longtable} %used for acronymns table
\usepackage{array} % for customising paragraph p columns alignment
\usepackage{tablefootnote} % obvious
\usepackage{arydshln} %for dotted v lines in tables
%
\usepackage{glossaries}
\usepackage{caption}
\usepackage{subcaption}
%
\usepackage{enumerate} %for lists
%
\usepackage{fancyhdr} %for custom page headers
%
\usepackage{imakeidx} % for indexing

%%% TIKZ Predefinitions
\usetikzlibrary{patterns}
\usetikzlibrary{decorations}
\tikzstyle{block} =
    [rectangle, draw
     % , fill=blue!20
      , text width=7.5em
      , text centered
      , node distance=3.5cm
      , rounded corners
      , minimum height=2em
      , scale =0.8]
\tikzstyle{blockL} =
    [rectangle, draw
     % , fill=blue!20
      , text width=7.5em
      , align=left
      , node distance=3.5cm
      , rounded corners
      , minimum height=2em
      , scale =0.8]
\tikzstyle{block20} =
    [rectangle, draw
     % , fill=blue!20
      , text width=9em
      , text centered
      , node distance=3.5cm
      , rounded corners
      , minimum height=2em
      , scale =0.8]
      \tikzstyle{block40} =
    [rectangle, draw
     % , fill=blue!20
      , text width=10.5em
      , text centered
      , node distance=3.5cm
      , rounded corners
      , minimum height=2em
      , scale =0.8]
 \tikzstyle{Vblock} =
    [rectangle, draw=none
     % , fill=blue!20
      , text width=15em
      , text centered
      , node distance=3.5cm
      , rounded corners
      , minimum height=2em
      , scale =0.8]
\tikzstyle{Lblock} =
    [rectangle, draw
     % , fill=blue!20
      , text width=35em
      , text centered
      , node distance=3.5cm
      , rounded corners
      , minimum height=2em
      , scale =0.8]
\tikzstyle{virtual} = [coordinate]
%%%%%%%%%%%%%%%%%%%%

\usepackage{sectsty}
\sectionfont{\centering}
\subsectionfont{\centering}
\usepackage{float} % for helping figures in twocol environ with [H]
\usepackage[font=small,labelfont=bf,textfont=it]{caption} % for caption text that
                                              % can be at all discerned
                                              % from normal text
 \newcommand{\fixme}[1]{{\color{red}#1}} 

\newenvironment{Figure}
  {\par\medskip\noindent\minipage{\linewidth}}
  {\endminipage\par\medskip}
\usepackage{lipsum}
\AtBeginDocument{\renewcommand{\bibname}{\centerline References}}
\begin{document}

{
	\centering
	\vspace{1cm}
	{\scshape\Huge\ UppSAT in the Cloud \par}
	\vspace{1cm}
	{\scshape\Large --- \par}
	\vspace{0.9cm}
	\begin{center}{Peter Backeman \& Albin Stjerna}\\
	Uppsala University, Uppsala, Sweden
	\end{center}
}

\begin{multicols}{2}

\section*{Abstract}

\textbf{ \textit { \fixme{We are doing something with some computers. That's a
      good idea because.} }}

\section*{Background}

SMT~solvers are used in different areas, with one important use case being
verification of hardware and software. For example, in model checking,
SMT~formulas are generated such that any solution corresponds to a hardware or
software bug.

Many research topics in SMT~solvers correspond to improving solver techniques
and devising new strategies and improved parameters of search. However, as the
problem is NP-hard, such improvements are typically a trade off with some cases
becoming faster and others slower. Therefore, when evaluating new methods
extensive benchmarking must be used.

These benchmarks are often run on local clusters or even local machines in
sequential order, leading to long waits from start until a complete picture of
the effects of the evaluated modifications are yielded. By instead using a cloud
approach, where not only the number of machines could be scaled, but also
dynamic load-balancing could be used, this waiting time could be greatly
reduced.

We therefore investigate how a SMT~solver can be adapted for benchmarking in a
cloud environment, with UppSAT~\cite{uppsat} as a case study. The produced
solution represents a partially automated testbed environment in the cloud with
a complete continuous integration pipeline from code pushed to Git to it being
benchmarked in the cloud, as well as a Representational State Transfer (REST)
API controlling the testbed environment. Moreover, a modular system of Docker
containers is used to package the solver for the test cases.

We argue that the use of Docker serves the dual purpose of enhancing both
repeatability and traceability of the experiments, as the entire experiment
environment is packaged into containers, allowing the use of Docker's tools for
comparison and analysis of containers in the event of unexpected results.
Additionally, as an added benefit, Docker ensures clean separation of
experiments.

Finally, we investigate how virtualisation affects measurement errors of
benchmarks, a topic in which there is a dearth of previous research.

\section*{Methods}

\fixme{\lipsum[100]}

\subsection*{System Design and Implementation}

\subsection*{Cloud Environment and Provisioning}

\section*{Results}

\fixme{\lipsum[100]}

\section*{Discussion}

\fixme{\lipsum[66]}

Also, something something, here is a reference~\cite{mell_nist_nodate}

\section*{Conclusion}

\fixme{\lipsum[100]}

\printbibliography

\end{multicols}
\end{document}
